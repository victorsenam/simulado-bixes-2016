% Configuração {{{
\documentclass[a4paper,11pt]{article}
\usepackage[brazil]{babel}
\usepackage[utf8]{inputenc}
%\usepackage[T1]{fontenc} % T2A para cirílico, acho
% \usepackage{arabtex} % para fonte arábica
% \usepackage{utf8}    % para fonte arábica

\usepackage{indentfirst}
\usepackage{float}
\usepackage{fancyvrb}
\usepackage{multicol}
\usepackage{amsmath}
\usepackage{amsfonts}
%\usepackage{hyperref}
\usepackage{epsfig}
\usepackage{multirow}
\usepackage{epigraph}

\usepackage{tikz}
\usetikzlibrary{arrows,shapes,positioning,topaths,intersections,calc,snakes}

\setlength{\marginparwidth}{0pt}
\setlength{\oddsidemargin}{-0.25cm}
\setlength{\evensidemargin}{-0.25cm}
\setlength{\marginparsep}{0pt}

\setlength{\textwidth}{16.5cm}
\setlength{\textheight}{25.5cm}

\setlength{\voffset}{-1in}

\newcommand{\PASTA}{.}

\newcommand{\insereArquivo}[1]{\VerbatimInput[xleftmargin=0mm,numbers=none,obeytabs=true]{\PASTA/#1}}

\newcommand{\memoriaProblema}[1]{\vspace{-0.3cm} \noindent { \emph{Memory Limit per test:} \texttt{#1 megabytes} \\}}

\newcommand{\tempoProblema}[1]{\vspace{-0.3cm} \noindent { \emph{Time Limit per test:} \texttt{#1 seconds} \\}}

\newcommand{\textoSaidaPadrao}{\vspace{0.2cm} \noindent \emph{A
saída deve ser escrita na saída padrão.}}

\newcommand{\textoEntradaPadrao}{\vspace{0.2cm} \noindent \emph{A
entrada deve ser lida da entrada padrão.}}


\newcommand{\exemplo}[2]{
\vspace{0.3cm}
\begin{minipage}[c]{0.9\textwidth}
\begin{center}
\begin{tabular}{|l|l|} \hline
\begin{minipage}[t]{0.5\textwidth}
\bf{Exemplo de entrada}
\insereArquivo{#1}
\vspace{.2cm}
\end{minipage}
&
\begin{minipage}[t]{0.5\textwidth}
\bf{Saída para o exemplo de entrada}
\insereArquivo{#2}
\end{minipage}\\
\hline
\end{tabular}
\end{center}
\end{minipage}}


\newcommand{\incluir}[2]{
\renewcommand{\PASTA}{#1}
\input{#1/#2}
}
% }}}


\begin{document}

% \setcode{utf8} % para fonte arábica

% Capa {{{
\begin{center}
\noindent

\hrule

\vspace{3.0cm}

\begin{Huge}
  % {\bf $1^{\underline{o}}$ Simulado para ingressantes 2015}
  {\bf II Simulado para Ingressantes}
\end{Huge}

\vspace{6.0cm}

\begin{Huge}
{\bf Caderno de Problemas}
\end{Huge}

\vspace{6.0cm}

\begin{Large}
	{\bf Universidade de São Paulo}
\end{Large}

\vfill


Quinta-feira, 28 de abril de 2016.

\vfill
\hrule
\end{center}

\thispagestyle{empty}

\newpage

\section*{Instruções}

\begin{itemize}
	\item A competição tem duração de 5 horas.
	\item A entrada de cada problema deve ser lida da entrada padrão (teclado).
	\item A saída de cada problema deve ser escrita na saída padrão (tela).
	\item Siga o formato apresentado na descrição da saída, caso contrário não é garantido que seu
	código será aceito.
	\item Na saída, toda linha deve terminar com o caracter `\texttt{\textbackslash n}'.
\end{itemize}

\section*{Pontuação} 

\begin{itemize}
	\item A cada problema aceito serão somados $x$ pontos a sua pontuação, onde $x$ é 
	o minuto de duração da prova onde problema foi submetido. Ex: Se você submeteu o 
	problema B aos 156 minutos de prova, então sua pontuação será acrescentada em 156
	pontos.
	\item Suponhamos que você mandou determinado problema $n$ vezes e apenas na vez $n+1$
	ele foi aceito. Então uma penalidade de 20 pontos por tentativa será acrescentado a sua
	pontuação, ou seja, será somado $n*20$ ao total já acumulado. Atenção: a penalização só 
	será adicionada se ao final das tentativas o problema for aceito.
	\item Você sempre ganha daqueles que resolveram menos problemas que você.
	\item O desempate entre equipes que resolveram o mesmo número de problemas é feito 
	pela pontuação: quem tiver a menor vence.
\end{itemize}

\vspace*{1.0cm}

\renewcommand{\arraystretch}{1.5}
\begin{center}
	\begin{tabular}{|c|l|p{5.8cm}|}
		\hline
		\multicolumn{2}{|c|}{Respostas das submissões}\\
		\hline
		Running on Test $\#$ & Seu programa está sendo testado no caso $\#$. Paciência.\\
		\hline
		Accepted & Código aceito. Parabéns!\\
		\hline
		Compilation error & Erro de compilação\\
		\hline
		Wrong answer & Errado. Pode tentar de novo.\\
		\hline
		Time limit exceeded & 
		Seu programa demora muito para dar a resposta (certa ou errada)\\
		\hline
		Runtime error & 
		Erro em tempo de execução (ex.: \textit{segmentation fault})\\
		\hline
	\end{tabular}
\end{center}
\renewcommand{\arraystretch}{1.0}

\newpage

\newcounter{letra}
\setcounter{letra}{1}
\newcommand{\proxLetra}{\Alph{letra}\stepcounter{letra}}

\incluir{../problemas/ambulatorio/docs}{enunciado.tex} \clearpage
\incluir{../problemas/aula/docs}{enunciado.tex} \clearpage
\incluir{../problemas/cinema/docs}{enunciado.tex} \clearpage
\incluir{../problemas/circular/docs}{enunciado.tex} \clearpage
\incluir{../problemas/restaurante/docs}{enunciado.tex} \clearpage
\incluir{../problemas/karaoke/docs}{enunciado.tex} \clearpage
\incluir{../problemas/kart/docs}{enunciado.tex} \clearpage
\incluir{../problemas/nim/docs}{enunciado.tex} \clearpage
\incluir{../problemas/pablito/docs}{enunciado.tex} \clearpage
\incluir{../problemas/perdido/docs}{enunciado.tex} \clearpage
\incluir{../problemas/raia/docs}{enunciado.tex} \clearpage
\incluir{../problemas/xadrez/docs}{enunciado.tex} \clearpage
\incluir{../problemas/japones/docs}{enunciado.tex} \clearpage

\end{document}
