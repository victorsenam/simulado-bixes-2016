% Configuração {{{
\documentclass[a4paper,11pt]{article}
\usepackage[brazil]{babel}
\usepackage[utf8]{inputenc}
%\usepackage[T1]{fontenc} % T2A para cirílico, acho
% \usepackage{arabtex} % para fonte arábica
% \usepackage{utf8}    % para fonte arábica

\usepackage{indentfirst}
\usepackage{float}
\usepackage{fancyvrb}
\usepackage{multicol}
\usepackage{amsmath}
\usepackage{amsfonts}
%\usepackage{hyperref}
\usepackage{epsfig}
\usepackage{multirow}

\usepackage{tikz}
\usetikzlibrary{arrows,shapes,positioning,topaths,intersections,calc,snakes}

\setlength{\marginparwidth}{0pt}
\setlength{\oddsidemargin}{-0.25cm}
\setlength{\evensidemargin}{-0.25cm}
\setlength{\marginparsep}{0pt}

\setlength{\textwidth}{16.5cm}
\setlength{\textheight}{25.5cm}

\setlength{\voffset}{-1in}

\newcommand{\PASTA}{.}

\newcommand{\insereArquivo}[1]{\VerbatimInput[xleftmargin=0mm,numbers=none,obeytabs=true]{\PASTA/#1}}

\newcommand{\textoDiversasInstancias}{A entrada é composta por
diversas instâncias. A primeira linha da entrada contém um inteiro~$T$
indicando o número de instâncias. }
\newcommand{\textoDiversasInstanciasEOF}{A entrada é composta por
  diversas instâncias e termina com final de arquivo (\texttt{EOF}).}

\newcommand{\arquivoProblema}[1]{\vspace{-0.3cm} \noindent {\em
Arquivo: \texttt{#1.[c|cpp|java|py]} \\}}

\newcommand{\textoSaidaPadrao}{\vspace{0.2cm} \noindent \emph{A
saída deve ser escrita na saída padrão.}}

\newcommand{\textoEntradaPadrao}{\vspace{0.2cm} \noindent \emph{A
entrada deve ser lida da entrada padrão.}}


\newcommand{\exemplo}[2]{
\vspace{0.3cm}
\begin{minipage}[c]{0.9\textwidth}
\begin{center}
\begin{tabular}{|l|l|} \hline
\begin{minipage}[t]{0.5\textwidth}
\bf{Exemplo de entrada}
\insereArquivo{#1}
\vspace{.2cm}
\end{minipage}
&
\begin{minipage}[t]{0.5\textwidth}
\bf{Saída para o exemplo de entrada}
\insereArquivo{#2}
\end{minipage}\\
\hline
\end{tabular}
\end{center}
\end{minipage}}


\newcommand{\incluir}[2]{
\renewcommand{\PASTA}{#1}
\input{#1/#2}
}
% }}}


\begin{document}

% \setcode{utf8} % para fonte arábica

% Capa {{{
\begin{center}
\noindent

\hrule

\vspace{3.0cm}

\begin{Huge}
  % {\bf $1^{\underline{o}}$ Simulado para ingressantes 2015}
  {\bf II Simulado para Ingressantes 2016}
\end{Huge}

\vspace{6.0cm}

\begin{Huge}
{\bf Caderno de Problemas}
\end{Huge}

\vspace{6.0cm}

\begin{Large}
	{\bf Universidade de São Paulo}
\end{Large}

\vfill


Quinta-feira, 28 de abril de 2016.

\vfill
\hrule
\end{center}

\thispagestyle{empty}

\newpage

\section*{Instruções}

\begin{itemize}
	\item A competição tem duração de 5 horas;
	\item Faltando 1 hora para o término da competição, o placar será congelado, ou seja, o placar 
	não será mais atualizado;
	\item Faltando 15 minutos para o término da competição, os times não receberão mais a resposta
	de suas submissões.
\end{itemize}

\vspace*{0.5cm}

\begin{itemize}
	\item A entrada de cada problema deve ser lida da entrada padrão (teclado);
	\item A saída de cada problema deve ser escrita na saída padrão (tela);
	\item Siga o formato apresentado na descrição da saída, caso contrário não é garantido que seu
	código será aceito;
	\item Na saída, toda linha deve terminar com o caracter `\texttt{\textbackslash n}';
	\item O nome do arquivo de códigos em Java deve ser \textbf{exatamente} como indicado abaixo
	do nome de cada problema. Para C/C++ é recomendado usar o nome indicado;
	\item Para códigos em Java, o nome da classe principal deve ser \textbf{igual} ao nome do
	arquivo.
\end{itemize}

\vspace*{1.0cm}

\renewcommand{\arraystretch}{1.5}
\begin{center}
	\begin{tabular}{|c|l|p{5.8cm}|}
		\hline
		\multicolumn{3}{|c|}{Respostas das submissões}\\
		\hline
		Not answered yet & \multicolumn{1}{|c|}{-} & Paciência\\
		\hline
		YES & \multicolumn{1}{|c|}{-} & Código aceito. Parabéns!\\
		\hline
		\multirow{11}{*}{NO} & Compilation error & Erro de compilação\\
		\cline{2-3}
		 & Wrong answer & Errado. Pode tentar de novo.\\
		\cline{2-3}
		 & \multirow{2}{*}{Time limit exceeded} & 
		Seu programa demora muito para dar a resposta (certa ou errada)\\
		\cline{2-3}
		 & \multirow{2}{*}{Runtime error} & 
		Erro em tempo de execução (ex.: \textit{segmentation fault})\\
		\cline{2-3}
		 & Problem name mismatch & Leia as duas últimas instruções\\
		\cline{2-3}
		 & \multirow{2}{*}{Presentation error} & Não está imprimindo no formato 
		exigido no enunciado\\
		\cline{2-3}
		 & \multirow{2}{*}{If possible, contact staff} & 
		Não sei, você conseguiu fazer algo inesperado\\
		\hline
	\end{tabular}
\end{center}
\renewcommand{\arraystretch}{1.0}



\newpage

\newcounter{letra}
\setcounter{letra}{1}
\newcommand{\proxLetra}{\Alph{letra}\stepcounter{letra}}

\incluir{../problemas/ambulatorio/docs}{enunciado.tex} \clearpage
\incluir{../problemas/aula/docs}{enunciado.tex} \clearpage
\incluir{../problemas/cinema/docs}{enunciado.tex} \clearpage
\incluir{../problemas/circular/docs}{enunciado.tex} \clearpage
\incluir{../problemas/japones/docs}{enunciado.tex} \clearpage
\incluir{../problemas/karaoke/docs}{enunciado.tex} \clearpage
\incluir{../problemas/kart/docs}{enunciado.tex} \clearpage
\incluir{../problemas/nim/docs}{enunciado.tex} \clearpage
\incluir{../problemas/pablito/docs}{enunciado.tex} \clearpage
\incluir{../problemas/perdido/docs}{enunciado.tex} \clearpage
\incluir{../problemas/raia/docs}{enunciado.tex} \clearpage
\incluir{../problemas/restaurante/docs}{enunciado.tex} \clearpage
\incluir{../problemas/xadrez/docs}{enunciado.tex} \clearpage

\end{document}
