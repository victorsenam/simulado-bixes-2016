\section*{Problema \proxLetra: MaratonIME vai ao karaokê}
\arquivoProblema{karaoke}

Depois de milênios repetindo o título do enunciado deste problema, sempre num tom animado e convidativo,
Nathan finalmente conseguiu convencer seus colegas a irem ao karaokê. Ele está simplesmente radiante com
esta conquista.

Mas há um problema. Depois de tanto ter insistido com seus amigos para irem ao karaokê, Nathan está com
medo de passar vergonha na hora de cantar os clássicos da música brasileira:
\begin{itemize}
    \item Garçon -- Reginaldo Rossi
    \item Boate azul -- Joaquim e Manuel
    \item Coração de papel -- Sérgio Reis
    \item Borbulhas de amor -- Fagner
    \item Você não me ensinou a te esquecer -- Fernando Mendes
\end{itemize}

Para evitar passar vexame, e para não desencorajar seus coleguinhas sobre futuras idas ao karaokê, Nathan
resolveu imprimir todas as cifras de músicas que estão disponíveis no karaokê, para poder consultar enquanto
canta. No entanto, isso gerou uma quantidade colossal de papel, que ele simplesmente não consegue carregar.

Mas a perseverança e engenhosidade de um programador com dor de cotovelo não é algo a ser subestimado.

Nathan reparou que, afinal de contas, só existem 7 notas musicais. O pessoal que entende do assunto costuma
representá-las pelas letras A,B,C,D,E,F e G. Mais ainda, é comum a mesma nota se repetir em sequência. Ele
resolveu então, comprimir as músicas, trocando toda ocorrência de notas repetidas pela nota e quantas vezes
ela ocorre.

Por exemplo, dado a sequência
\[(A,A,A,B,B,B,C,G,G,G,G,G,G,G,G,G,G,G)\]
a versão comprimida é 
\[A3B3C1G11\]

Infelizmente, Nathan também precisa arrumar seu terno florido e pentear sua barba -- dois trabalhos homéricos -- e
ficou sem tempo para comprimir as notas. Ajude-o a não passar vergonha, escrevendo um programa que faça
este serviço.

\subsection*{Entrada}

Cada entrada consiste de apenas uma linha, uma sequência de caractéres $S$ tal que $|S| \leq 10^5$, formada apenas
pelas letras A,B,C,D,E,F e G.

\subsection*{Saída}
Para cada entrada, imprima uma única linha, tal que cada sequência de notas iguais seja substituída pela nota que
ocorre e quantas vezes ela ocorre, conforme o exemplo.

% \textoSaidaPadrao

\subsection*{Restrições}
\begin{itemize}
  \item $|S| \leq 10^5$
\end{itemize}

\subsection*{Exemplos}

\exemplo{tests/in1}{tests/out1}
