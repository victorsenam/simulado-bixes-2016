\section*{Problema \proxLetra: MaratonIME joga Xadrez}
\arquivoProblema{xadrez}

Nosso querido Nathan, quando guri, gostava muito de jogar Xadrez, mas isso já faz muito tempo. Um dia desses ele foi desafiado pelo famigerado @luisgust para uma partida de xadrez e, como gosta muito de um desafio, aceitou. O problema é que Nathan anda se esquecendo de algumas das regras do jogo, então ele pediu sua ajuda para determinar, em alguns momentos do jogo, se uma dada peça inimiga pode ser capturada com um único movimento. \\
O xadrez, no MaratonIME, é representado como um grid de caracteres. Ao invés de jogar com peças brancas e pretas, os times jogam com peças maiúsculas (representadas por caracteres maiúsculos) e minúsculas (represetadas por caracteres minúsculos). O Nathan escolheu jogar com as peças minúsculas, porque elas jogam antes. \\
Além disso, usualmente, as posições de um tabuleiro de xadrez são dadas num sistema de coordenadas específico. Toda posição é um par com um caractere entre \texttt{a} (inclusive) e \texttt{h} (inclusive), representando a coluna, e um inteiro entre \texttt{1} (inclusive) e \texttt{8} (inclusive), representando a linha. Por exemplo, a posição \texttt{d2} se refere à quarta casa (da esquerda para a direita) da segunda linha coluna (de cima para baixo) e a posição \texttt{f6} se refere à sexta casa da sexta linha. As peças minúsculas começam do lado de baixo do tabuleiro, ou seja, nas linhas \texttt{7} e \texttt{8}. \\
Além disso, o termo posição adjacente de uma dada posição se refere a qualquer uma das posições que compartilham pelo menos um vértice com ela, isto é, duas posições são adjacentes se e somente se o máximo entre a distância entre as linhas e a distância entre as colunas das duas posições for igual a 1. Por exemplo, a posição \texttt{c4} é adjacente a 7 posições: \texttt{b3}, \texttt{b4}, \texttt{b5}, \texttt{c3}, \texttt{c5}, \texttt{d3}, \texttt{d4} e \texttt{d5}. \\
O MaratonIME, além disso, usa um sistema um pouco simplificado de movimentos de xadrez. Para te ajudar com o seu programa, foi fornecido um manual com os movimentos que cada peça pode realizar para capturar outra peça e as letras que representam cada peça. \\
\begin{itemize}
    \item O peão, representado pelos caracteres \texttt{p} ou \texttt{P}, pode capturar peças adjacentes a ela que compartilhem uma diagonal com ela e estejam à frente dela, ou seja, um peão minúsculo pode capturar uma peça adjacente a ele que tenha uma diagonal em comum com ele e que esteja em uma linha estritamente menor que a dele, por exemplo, um peão na posição \texttt{c4} pode capturar peças nas posições \texttt{b3} ou \texttt{d3}.
    \item O cavalo, representado por \texttt{c} ou \texttt{C}, faz movimentos em L em qualquer um dos 8 sentidos possíveis, ou seja, ele anda duas casas em uma direção qualquer e, depois, uma casa em uma direção perpendicular à primeira. Um cavalo na posição \texttt{c4} pode capturar, em um movimento, peças nas posições \texttt{a3}, \texttt{a5}, \texttt{b2}, \texttt{b6}, \texttt{d2}, \texttt{d6}, \texttt{e3} e \texttt{e5}.
    \item A torre, representada por \texttt{t} ou \texttt{T}, consegue capturar qualquer peça que tenha alguma coordenada igual a ela e que seja visível por ela em uma linha reta, ou seja, não existe nenhuma peça na linha entre a torre e a peça a ser capturada. Por exemplo, uma torre em \texttt{c4} pode capturar qualquer peça na coluna \texttt{c} e na linha \texttt{4}, desde que não haja nenhuma peça entre elas. Se houver uma torre em \texttt{c4} e uma outra peça qualquer em \texttt{c6}, a torre não pode capturar ninguém em \texttt{c7} nem em \texttt{c8}, mas uma torre em \texttt{c4} pode capturar uma peça em \texttt{a4} desde que não tenha ninguém em \texttt{b4}.
    \item O bispo, representado por \texttt{b} ou \texttt{B}, consegue capturar qualquer peça que esteja em alguma das diagonais que passam pelo bispo e que seja visivel por ele em uma linha reta, ou seja, uma peça tal que a diferença entre as coordenadas-linha dela e do bispo e as coordenadas-coluna dela e do bispo sejam iguais. Por exemplo, um bispo em \texttt{c4} consegue capturar alguém em \texttt{f7} desde que não haja ninguém nem em \texttt{d5} e em \texttt{e6}.
    \item A rainha, representada por \texttt{r} ou \texttt{R}, consegue fazer os movimentos da torre e os movimentos do bispo, ou seja, consegue, em um movimento, capturar qualquer peça que tenha alguma coordenada igual à dela ou que esteja em alguma diagonal a qual ela pertence desde que não haja nenhum obstáculo (outra peça) na linha reta entre a rainha e a peça que se quer capturar.
    \item O rei consegue capturar qualquer peça que esteja em uma casa adjacente à dele, seguindo a definição de adjacente já explicada.
    \item O caractere \texttt{.} representa uma posição vazia.
\end{itemize}
Seu programa deve receber uma matriz e a posição da peça inimiga que o Nathan quer capturar e responder \texttt{Sim} se é possível capturar essa peça com um movimento e \texttt{Não} caso contrário. \\
É garatido que existe sempre uma peça inimiga na posição dada, é garantido também que cada time contém, no máximo, dois bispos, duas torres, dois cavalos, 8 peões, um rei e uma rainha. \\


\subsection*{Entrada}
\textoDiversasInstanciasEOF

Cada instância consiste de 8 linhas com 8 caracteres cada. A primeira linha contém os caracteres nas posições \texttt{a1}, \texttt{b1}, \dots, \texttt{h1}. A segunda linha contém os caracteres nas posições \texttt{a2}, \texttt{b2}, \dots, \texttt{h2}. E assim por diante. Depois da matriz, em uma linha separada, são dadas as coordenadas que indicam a posição da peça inimiga que deve ser capturada. \\


\subsection*{Saída}

A saída consiste em uma única linha para cada instância contendo a string \texttt{Sim} se é possível capturar a peça ou \texttt{Nao} caso contrário.

% \textoSaidaPadrao

\subsection*{Exemplos}

\exemplo{tests/in1}{tests/out1}
