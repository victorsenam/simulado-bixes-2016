\section*{Problema \proxLetra: MaratonIME vai à aula (ou não)}
\arquivoProblema{aula}

No MaratonIME, assim como em muitos outros grupos, alguns alunos querem apenas comparecer a aulas o suficiente para não reprovarem por faltas (como sabemos, na USP é necessário 70\% de presença), porém outros são dedicados e tentam conseguir a maior presença possível, indo para a faculdade mesmo quando estão doentes ou incapacitados. Curiosamente não existem outros tipos de alunos no MaratonIME.

Madeira, um antigo membro do MaratonIME, precisa de ajuda, ele está cursando a matéria MAC4815162342, e compareceu a $k$ das $m$ aulas que já foram ministradas, sendo que MAC4815162342 tem $n$ aulas totais por semestre. Ele te pediu ajuda para descobrir como melhor cumprir seus objetivos, mas como você é novo no MaratonIME, não sabe que tipo de aluno ele é. Com vergonha de perguntar mais, você decide resolver os dois problemas, assim não há como errar.


\subsection*{Entrada}
\textoDiversasInstanciasEOF

A única linha de entrada possui três inteiros $n$, $m$ e $k$.

$n$ é o número de aulas de MAC4815162342 por semestre, $m$ é quantas dessas aulas já foram dadas e $k$ a quantas aulas Madeira compareceu.

\subsection*{Saída}

Na primeira linha imprima o número mínimo de aulas que Madeira precisa comparecer para conseguir \emph{pelo menos} 70\% de presença, ou $-1$ se for impossível conseguir 70\% de presença.

Na segunda linha imprima a maior porcentagem de presença que Madeira pode conseguir, se for para todas as aulas a partir da próxima. Esse valor deve estar \emph{arredondado para baixo} para o inteiro mais próximo. Não imprima '\%'.

% \textoSaidaPadrao

\subsection*{Restrições}

\begin{itemize}
  \item $1 \leq n \leq 10^7$
  \item $0 \leq k \leq m \leq m$
\end{itemize}

\subsection*{Exemplos}

\exemplo{tests/in1}{tests/out1}

\subsection*{Notas}

No segundo exemplo, a porcentagem máxima que Madeira pode conseguir é $90.9090$.
