\section*{Problema \proxLetra: MaratonIME joga Nim}
\memoriaProblema{256}
\tempoProblema{1}

Você abre seus olhos, mas tudo continua escuro. O mundo está escuro, e tudo balança. Você percebe que está trancado, mas antes de começar a se desesperar, você ouve um porta abrindo e a luz invade sua visão e te cega por alguns momentos.

Te ajudam a sair, você estava trancado num porta-malas. Você não reconhece as faces mascaradas, mas se lembra que no último treino para Maratona te falaram que ``o começo ainda estava por vir''. ``Então esse é o lendário desafio de iniciação do MaratonIME'', você já tinha ouvido boatos desse evento, e se sente honrado de ter sido selecionado.

Depois de entrar em um prédio abandonado, te sentam em uma cadeira velha. O primeiro teste é assistir um jogo de futebol sem nenhuma reação de animação. Fácil. O segundo é instalar Linux em um notebook em menos de 5 minutos. Você já estava preparado, sempre carrega seu pendrive com ArchLinux em caso de emergência. Você enfrenta mais testes, e consegue passar em todos apesar de algumas dificuldades.

Depois de horas, os integrantes tiram suas máscaras, e cada um retira uma moeda do bolso. ``Ganhei! E ainda fiquei rico'' você pensa, mas percebe que eles colocam as moedas numa mesa na sua frente, divididas em duas pilhas. Renzo, o grande chefe do MaratonIME, pega uma cadeira e senta na sua frente. Vocês devem jogar uma partida de Nim, e se você vencer se torna um membro honorário do MaratonIME, ou seja, ganha um balão.

Nim é um jogo de dois jogadores, que alternam turnos. Duas pilhas de moedas ficam em uma mesa e em uma jogada você pode tirar qualquer quantidade não nula de moedas de uma das pilhas. O último jogador a fazer uma jogada (e deixar as duas pilhas vazias) ganha.

Você começa o jogo. Para não ser injusto foi garantido que existe uma forma de você ganhar. Escreva um programa que vença Renzo 100\% das vezes.


\subsection*{Entrada}

Na primeira linha, dois interos, $x$ e $y$, o tamanho das pilhas, com $0 \leq x, y \leq 10^4$. É garantido que é possível vencer esse jogo.

\subsection*{Interação}

Na sua jogada, imprima dois inteiros, $i$ e $x$, onde $i$ é o número da pilha que você vai tirar moedas ($i \in \{1, 2\}$), e $x$ o número de moedas que vai retirar ($x \geq 1$, tal que a pilha $i$ tenha pelo menos $x$ pedras).

Na jogada do Renzo, leia dois inteiros, no mesmo formato que os da sua jogada. Toda jogada do Renzo vai ser válida.


% \textoSaidaPadrao

\subsection*{Exemplos}

\exemplo{tests/in1}{tests/out1}

\subsection*{Notas}

É claro que não fazemos um desafio de iniciação assim :P

No exemplo, temos uma pilha com duas moedas e outra com uma. Você tira uma moeda da primeira pilha, agora qualquer moeda que Renzo retirar, você pode retirar a outra e ganhar.

Lembre-se de, depois de imprimir sua jogada, fazer a descarga da saída, como \texttt{fflush(stdout);} em C, \texttt{cout.flush();} em C++, ou \texttt{sys.stdout.flush()} em Python.
