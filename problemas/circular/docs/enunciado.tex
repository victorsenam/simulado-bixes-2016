\section*{Problema \proxLetra: MaratonIME faz alguma coisa}
\arquivoProblema{Exemplo}

Dois pássarinho voavam pelo lindo céu azul da Austrália quando encontraram um dos membros do MaratonIME preso em uma árvore. Ele gritava ``Me ajudem, ganhei tantos balões durante o simulado que comecei a voar e acabei preso aqui, vocês poderiam me ajudar a descer?''. Os pássaros pensaram e refletiram quais opções eles tinham naquele estranho momento:

\begin{itemize}
	\item ``Podemos rir e ir embora, mas ele pode tentar agredir a gente com um balão.''
	\item ``Podemos ajudar e pedir comida em troca (além de metade de seus balões).''
\end{itemize}

Após muitas considerações, os pássaros decidiram ajudar o pobre maratonista, mas havia um problema: será que eles conseguiriam suportar o peso do rapaz? Sua função é dado o peso do competidor e quanto cada pássaro consegue carregar, dizer se eles conseguem trazer o maratonista para a terra. Os pássaros conseguem trazer o competidor se as somas dos valores que cada um consegue carregar for maior ou igual que o seu peso. 


\subsection*{Entrada}
\textoDiversasInstanciasEOF

A entrada consiste em uma única linha contendo um inteiro $p$, o peso do rapaz, seguido de dois inteiros $x$ e $y$, o quanto cada pássaro consegue carregar.


\subsection*{Saída}

A saída consiste em uma única linha contendo a string ``Sim'' se é possível salvar o rapaz, ou ``Nao'' caso contrário.

% \textoSaidaPadrao

\subsection*{Restrições}
\begin{itemize}
  \item $0 \leq p, x, y \leq 10^9$
\end{itemize}

\subsection*{Exemplos}

\exemplo{tests/in1}{tests/out1}
