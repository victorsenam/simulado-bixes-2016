\section*{Problema \proxLetra: MaratonIME vai ao restaurante japonês}
\memoriaProblema{256}
\tempoProblema{1}

Nathan e Yan são programadores de todo coração. Eles aplicam seus
conhecimentos de algoritmos em contextos que transcendem a maratona,
otimizando as coisas mais triviais do dia a dia.

Alguns questionam se eles só não são meio malucos mesmos, e se não 
era melhor apenas fazer o que precisava ser feito.

De qualquer forma, eventualmente rolam conflitos quando as abordagens
deles divergem sobre o que deve ser feito. Isso rola quando eles vão
em restaurantes japoneses.

Todo mundo sabe que o objetivo, num rodízio, é comer a maior quantidade
de comidas distintas. Nathan e Yan diferem na abordagem. Cada um aplica
um algoritmo e diz que o dele é o que garante que haverá mais iguarias
diferentes sendo devoradas.

Nathan sempre come o prato que estiver mais fresco na mesa, ou seja,
o que chegou mais recentemente. Em caso de empate, ele prefere o prato
que ele comeria mais rápido.

Yan respeita metodicamente a ordem de chegada dos alimentos, e, em
caso de empate, também prefere o que ele comeria mais rápido.

Dado o momento de chegada dos pratos, quanto tempo cada um leva pra comer,
e quanto tempo eles ficarão no restaurante, diga qual dos dois vai degustar
mais pratos diferentes, ou se haverá empate.
\subsection*{Entrada}

A primeira linha da entrada tem dois inteiros: $0 < n \leq 10^5$, que representa
quantos pratos serão servidos, e $0 < T \leq 10^6$, que representa quantos
minutos eles vão ficar no restaurante.

Depois seguem $n$ linhas, cada uma com dois inteiros $t_i$ e $c_i$, que
representam em minutos, respectivamente, o tempo que demora pra comer 
aquele prato e o momento da chegada dele.

Você pode assumir que todos os $c_i$ serão dados em ordem não decrescente, ou
seja, que para todo $i$ vale $c_i \leq c_{i+1}$.

\subsection*{Saída}

Imprima ``Nathan'', ``Yan'' ou ``Empate'', respondendo a pergunta sobre
pratos distintos.

% \textoSaidaPadrao

\subsection*{Restrições}
\begin{itemize}
    \item Para todo $i$, vale que $0 \leq c_i \leq 100$
    \item Para todo $i$, vale que $0 \leq t_i \leq 10^6$
\end{itemize}

\subsection*{Exemplos}
\exemplo{tests/in1}{tests/out1}

\subsection*{Notas}
Repare que o atendimento em alguns restaurantes é muito ruim, e a comida
pode acabar só chegando depois que o MaratonIME já foi embora de barriga
cheia.
