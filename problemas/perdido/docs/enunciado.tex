\section*{Problema \proxLetra: MaratonIME volta para o alojamento}
\arquivoProblema{volta}

Muito frequentado pelo maratonistas mais tradicionais, o acampamento de verão é muito conhecido por suas festas nababescas. Contudo após as festas os competidores sempre devem voltar para seus respectivos alojamentos. Essa volta nem sempre é tranquila, os competidores não muito sóbrios acabam andando em zig-zague e muitas vezes perdendo ou achando dinheiro, além de serem roubados no caminho. O MaratonIME entretanto sempre tem um membro sóbrio em seu grupo, o que garante que eles não vão perder dinheiro no caminho a não ser que sejam roubados. Voltando para o alojamento o MaratonIME sabe que podem ligar para seu coach, Renzo, que irá buscá-los instantaneamente. Entretanto, eles querem a sua ajuda para escrever um programa que, sabendo o caminho, diga qual é o Máximo de dinheiro que é possível levar para o alojamento.\\
\\
Então sua tarefa será escrever um programa que dado um grid N por M e sabendo que os membros do MaratonIME andam em zig-zague quando bêbados diga qual é o máximo de dinheiro que é possível levar para o alojamento. Deve se levar em consideração os seguintes fatos:\\
\begin{enumerate}
    \item Os membros do MaratonIME saem da festa e chegam na rua pelo ponto superior esquerdo e começam andando para a direita.
    \item Os membros do MaratonIME saem completamente sem dinheiro da festa.
    \item Sempre ao chegar ao fim da rua eles descem um ponto e andam na direção contrária.
    \item Sempre que encontram um ladrão no caminho, este lhes roubam todo o dinheiro acumulado.
    \item Eles podem ligar para Renzo a qualquer momento, e Renzo irá busca-los instantaneamente.
\end{enumerate}
\subsection*{Entrada}

A enxtrada consiste de dois inteiros, N e M, indicando respectivamente o tamanho da rua e a largura dela, seguidos por N linhas contendo M caracteres que podem ser:\\
\begin{itemize}
    \item \texttt{'\_'} indicando que não há nada naquele ponto da rua.
    \item \texttt{'.'} indicando que há uma moeda de um real naquele ponto da rua.
    \item \texttt{'L'} indicando que há um ladrão naquele ponto da rua.
\end{itemize}

\subsection*{Saída}

A saída consiste de um único inteiro indicando o máximo de dinheiro que MaratonIME pode trazer para o alojamento.\\
% \textoSaidaPadrao

\subsection*{Restrições}
\begin{itemize}
  \item $1 \leq N,M \leq 10^3 $
\end{itemize}

\subsection*{Exemplos}

\exemplo{tests/in1}{tests/out1}
