\section*{Problema \proxLetra: MaratonIME vai ao cinema}
\arquivoProblema{cinema}

Acredite se quiser, estudos indicam que não é apenas de maratona que vive o MaratonIME. 

Num sábado, depois de simular uma prova, nossa turma de heróis resolveu ir aproveitar
a tarde numa sessão de cinema. Mas eles não escolheram ir a um cinema qualquer, mas sim
ir assistir um filme $n$-D, ou seja, com $n$ dimensões.

Em certa cena do filme, durante uma perseguição, o personagem principal pulou por cima
de uma corrente de estacionamento. É óbvio que no filme este ato foi feito de forma 
casual e atlética, mas isto deixou Yan furioso.

``Hollywood fica fazendo essas coisas parecerem fáceis, mas é muito difícil pular por
cima de uma corrente!''

O resto da turma, após o filme, argumentou com Yan que o ator com certeza pulou por cima
da corrente, e este discordou, afirmando que só podia ser um truque de câmera,ou efeitos
especiais.

Para provar seu ponto, ele comprou outro ingresso para o filme e levou instrumentos de
medição muito precisos para a sala. O plano de Yan era mostrar que a distância do pé do
ator ao chão era menor que a distância da corrente ao chão, o que provaria que o ator
de fato não pulou a corrente.

Mas as contas em $n$ dimensões são complicadas. Todo mundo sabe que a distância no plano
entre dois pontos $(x_0,y_0)$ e  $(x_1,y_1)$ é dada por
\[ \sqrt{(x_0-x_1)^2 + (y_0-y_1)^2}\]
Muita gente também sabe que a distância entre dois pontos $(x_0,y_0,z_0)$ e $(x_1,y_1,z_1)$
no espaço tridimensional é calculada pela fórmula
\[\sqrt{(x_0-x_1)^2 + (y_0-y_1)^2 + (z_0-z_1)^2}\]

Ambas as fórmulas descrevem a chamada distância euclideana, no caso bi e tri-dimensional.
Seu trabalho aqui é, dado três pontos em $n$ dimensões, dizer se o par mais próximo é o
entre o pé e o chão, ou entre o chão e a corrente, de acordo com a distância euclideana.

\subsection*{Entrada}
\textoDiversasInstanciasEOF

A entrada começa com um inteiro $n$, $0 < n \leq 10^5$, e depois três linhas, cada uma
representando um ponto num espaço $n$-dimensional.

A segunda  linha representa as coordenadas do \textbf{chão}, e consiste de $n$ inteiros $a_i$.

A terceira linha representa as coordenadas do \textbf{pé do mocinho}, e consiste de $n$ inteiros $b_i$.

A quarta   linha representa as coordenadas da \textbf{corrente}, e consiste de $n$ inteiros $c_i$.

\subsection*{Saída}

Imprima quem está certo nesta história toda, ou seja, imprima ``Yan'' se a distância do
chão pro pé do mocinho é \textbf{menor ou igual} a distância do chão pra corrente, ou
imprima ``MaratonIME'' se a corrente estiver mais próxima.

% \textoSaidaPadrao

\subsection*{Restrições}
\begin{itemize}
  \item Para todo $i$, $|a_i| \leq 10^4$
  \item Para todo $i$, $|b_i| \leq 10^4$
  \item Para todo $i$, $|c_i| \leq 10^4$
\end{itemize}

\subsection*{Exemplos}

\begin{center}
\exemplo{tests/in1}{tests/out1}

\exemplo{tests/in2}{tests/out2}
\end{center}
