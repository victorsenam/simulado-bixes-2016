\section*{Problema \proxLetra: MaratonIME ajuda Pablito}
\arquivoProblema{pablito}

Como toda pessoa culta e letrada sabe, os ratos são os seres mais inteligentes
do planeta. Os golfinhos são os segundos.

O MaratonIME sabe da hierarquia entre as espécies, e a utiliza em seu favor.
Em geral, quando eles precisam de recurso, eles sabem que é sempre bom ter um
inteligente rato auxiliando. Mas os ratos não sentem muita empatia por nós,
primatas, e só nos ajudam se eles nos devem favores.

Assim, o MaratonIME decidiu ajudar o ratinho Pablito. Pablito está estudando a
genealogia dos ratos, para facilitar clonagem e mapeamento genômico. Felizmente,
parte do trabalho já está feita pela forma com a qual os ratos se identificam.

A sociedade dos ratos, historicamente, se organiza de forma matriarcal. No princípio,
cada Matriarca foi identificada com um número que só é divisível por 1 e por ele mesmo.
Era a chamada era primordial, dos primeiros ratos, primário da civilização.

Os demais ratos são sempre descendentes dessa linhagem. Assim, cada um deles é
identificado com o produto da identidade de seus pais, e tudo está bem definido.

Pablito precisa, dado dois ratos, saber se eles tem algum descendente em comum.
Sua única ferramenta é número de identificação de cada ratinho, que é sempre um
inteiro positiva com no máximo 16 dígitos.

Crie um programa que diz se o par de ratos que tem aquela identificação tem algum
antecessor em comum.

\subsection*{Entrada}
\textoDiversasInstanciasEOF

A entrada começa com um inteiro positivo $t \leq 10^6$.

Depois seguem $t$ linhas, cada uma com dois inteiros de identificação dos ratinhos.
\subsection*{Saída}

Para cada linha, responda ``Sim'' se os ratinhos tem um parente em comum,
e ``Não'', caso contrário.

% \textoSaidaPadrao

\subsection*{Exemplos}
\exemplo{tests/in1}{tests/out1}
