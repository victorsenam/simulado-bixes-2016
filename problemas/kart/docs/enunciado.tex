\section*{Problema \proxLetra: MaratonIME vai ao kart}
\memoriaProblema{256}
\tempoProblema{1}

Certo dia após um contest, os maratonistas sentiam-se cabisbaixos por conta de resultados ruins. Vendo a situação, Renzo sugeriu que fizessem algo divertido para descontrair a cabeça. Após discussão fervorosa, decidiram correr de kart. Na procura de um kartódromo que fosse viável para todos, acharam o Kartforces, um kartódromo próximo à Cidade Universitária. Entretanto, a pista era muito pequena e comportava apenas dois corredores por bateria. Como bons maratonistas apaixonados por competição, montaram um torneio justo onde todos corriam contra todos, dois a dois, uma única vez. Em cada corrida, o vencedor somava um ponto no placar. Empates são admitidos. O grande campeão foi o maior pontuador. Sabe-se que $n$ maratonistas estavam presentes e:

\begin{itemize}
	\item Cada maratonista possui uma habilidade $h_i$.
	\item Se $h_i > h_j$ onde $1 \leq i,j \leq n$ e $i \neq j$, então o maratonista $i$ vence a corrida contra o maratonista $j$.
\end{itemize}

Você teve acesso a habilidade de todos os maratonistas e agora se pergunta quem foi o grande campeão.

\subsection*{Entrada}

Cada instância é composta por duas linhas. 
A primeira linha contém um inteiro $n$, o número de maratonistas.
A segunda linha contém $n$ inteiros $h_i$, a habilidade do $i$-ésimo maratonista. 


\subsection*{Saída}

A saída consiste em um único inteiro $i$, o maratonista campeão. Caso não seja possível determinar o campeão, imprima $-1$.

% \textoSaidaPadrao

\subsection*{Restrições}
\begin{itemize}
  \item $1 \leq n \leq 10^5$
  \item Para todo $i$, $0 \leq h_i \leq 10^9$
\end{itemize}

\subsection*{Exemplos}

\begin{center}
\exemplo{tests/in1}{tests/out1}
\exemplo{tests/in2}{tests/out2}
\end{center}
