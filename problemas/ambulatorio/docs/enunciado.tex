\section*{Problema \proxLetra: MaratonIME joga Cîrokime}
\memoriaProblema{256}
\tempoProblema{1}

Os membros do MaratonIME adoram se divertir, gostam tanto que inventaram um jogo e o nomearam ``Cîrokime''. O jogo funciona da seguinte forma:\newline 
Primeiro $n$ copos com \textit{Cîroc\footnote{Cîroc é uma marca de vodka feita na França, conhecida por seu alto preço de venda no mercado.}} são enfileirados em uma linha, à frente de cada copo $i$ é escrito um número $a_i$. É garantido que, para todo $ 1 \leq i < n$ temos que $a_i < a_(i + 1)$. Em seguida os números são cobertos e o jogo começa.
O jogador deve então achar o copo que possui um determinado número $x$, é garatindo que tal copo existe. Para isso ele deve escolher um copo $i$ e beber a bebida, em seguida o seu número $a_i$ é revelado e se for igual ao valor $x$ o jogo encerra, caso contrário, o jogador deve escolher outro copo e assim por diante. 

``Cîrokime'' é uma tradição entre os membros do MaratonIME, eles jogam em toda festa\footnote{Balalaika serve.}. Na última festa, Sussu ficou em último lugar, teve que beber os $n$ copos e só no fim encontrou o copo certo. Além de ter ficado em último lugar, Sussu passou mal por ter bebido tanto e teve que ser carregado para casa\footnote{História baseada em fatos reais.}. Mas a próxima festa está marcada e Sussu quer recuperar sua dignidade, para isso ele quer saber, no pior caso, qual o número máximo de copos que ele terá que beber se jogar de forma ótima.

\subsection*{Entrada}

A entrada consiste em duas linhas. Na primeira, um inteiro $n$, o número de copos. A segunda contém $n$ inteiros $a_i$ com $1 \leq i \leq n$, em ordem crescente, os valores escondidos em cada copo.


\subsection*{Saída}

A saída consiste em uma única linha contendo um único inteiro: o número mínimo de copos que Sussu deve beber no pior caso se jogar de forma ótima.

% \textoSaidaPadrao

\subsection*{Restrições}
\begin{itemize}
  \item $1 \leq n \leq 10^5$
  \item Para todo $i$, $1 \leq a_i \leq 10^9$
\end{itemize}

\subsection*{Exemplos}

\begin{center}
\exemplo{tests/in1}{tests/out1}
\exemplo{tests/in2}{tests/out2}
\end{center}

